\section{Related Work}
\label{sec:related}

The original SCTP motivation and protocol model are well documented in standards and reference texts \cite{rfc9260,stewart2001sctp}. The socket API formalization in RFC 6458 is directly relevant to runtime and language binding design \cite{rfc6458}. Extensions such as partial reliability and stream scheduling further differentiate SCTP from TCP-centric abstractions \cite{rfc3758,rfc8260}.

Several prior studies evaluated SCTP capabilities. Iyengar et al. explored concurrent multipath transfer over SCTP multihoming \cite{iyengar2006cmt}. Penoff et al. described a portable userspace SCTP stack and portability/performance trade-offs \cite{penoff2012userspacesctp}. These works support the view that SCTP is both implementable and practically useful across deployment contexts.

More generally, transport evolution research has shown how deployment friction can ossify protocol diversity \cite{honda2011extendtcp}. This paper contributes from a runtime-engineering perspective: language-level first-class APIs materially reduce that friction by making non-TCP transports available through familiar interfaces.

Finally, WebRTC data channels demonstrate broad production exposure of SCTP semantics in modern applications \cite{rfc8831,rfc8832}. This real-world adoption weakens arguments that SCTP is niche or obsolete.
