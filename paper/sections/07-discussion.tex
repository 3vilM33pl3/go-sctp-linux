\section{Discussion}
\label{sec:discussion}

\subsection{Why This Supports ``First-Class'' Status}
The implementation and evaluation provide concrete evidence for first-class SCTP treatment in a modern runtime:

\begin{itemize}
\item \textbf{API symmetry:} SCTP follows the same naming and usage patterns as TCP/UDP in Go's \texttt{net} package.
\item \textbf{Internal symmetry:} SCTP plugs into existing resolver, dial, listen, and file-descriptor workflows.
\item \textbf{Interoperability:} wire-level behavior is compatible with independent Linux C++ SCTP endpoints.
\item \textbf{Operational symmetry:} tests and harness execution integrate into standard CI-style workflows.
\end{itemize}

This is precisely what ``first-class'' should mean in a runtime context.

\subsection{The Extensibility Argument}
Internet architecture intentionally supports protocol evolution \cite{cerf1974protocol,rfc1122,rfc1958}. Treating SCTP as a default-capable option is consistent with this model and avoids unnecessary application-layer reimplementation of transport semantics. This is aligned with end-to-end design reasoning: semantics that belong at transport should not be rebuilt repeatedly in each application \cite{saltzer1984endtoend}. In practice, forcing message-oriented or multi-stream designs onto TCP often reintroduces complexity that SCTP already standardizes.

\subsection{Additional Practical Arguments}
Beyond architectural correctness, several practical arguments support broader SCTP exposure:
\begin{itemize}
\item \textbf{Correctness:} protocol-level message framing reduces application parsing ambiguity.
\item \textbf{Performance isolation:} multistreaming can reduce cross-flow head-of-line coupling.
\item \textbf{Resilience:} multihoming and association semantics improve failover options.
\item \textbf{Security posture:} SCTP's association handshake and extension mechanisms provide robust baseline behavior \cite{rfc9260}.
\end{itemize}

None of these claims require replacing TCP/UDP; they require giving developers a native, standards-based choice.
