\section{Introduction}
\label{sec:intro}

Transport protocol choice in many software stacks has narrowed to TCP or UDP, even though Internet architecture never required such a binary choice. The original internetworking design emphasized extensibility and evolution at protocol boundaries \cite{cerf1974protocol,rfc1122,rfc1958}. SCTP was standardized to provide reliable, congestion-controlled, message-oriented transport with features that are difficult or expensive to emulate correctly above TCP or UDP \cite{rfc9260,rfc6458,stewart2001sctp}.

This paper revisits an earlier 2013 implementation effort and updates the work for Linux and a modern Go runtime source tree. The goal is not to present SCTP as ``new'', but to demonstrate that it can be integrated into a mainstream language runtime with the same first-class treatment given to TCP and UDP. In this context, first-class means:

\begin{itemize}
\item explicit address and connection types in the standard networking package,
\item normal resolver, dial, and listen flows,
\item direct support for protocol-specific metadata and controls,
\item test coverage and interoperability validation against an independent implementation.
\end{itemize}

The implementation in this repository adds SCTP support directly inside Go's \texttt{net} package on Linux, including public API surface (\texttt{SCTPAddr}, \texttt{SCTPConn}, \texttt{DialSCTP}, \texttt{ListenSCTP}), Linux data-path support for \texttt{sendmsg}/\texttt{recvmsg}, and targeted test and interop harnesses.

The paper's central thesis is that SCTP belongs next to TCP and UDP in modern runtime APIs because:

\begin{enumerate}
\item Internet protocol architecture is intentionally extensible \cite{rfc1122,rfc1958}.
\item SCTP is a mature standards-track transport with broad, production-relevant semantics \cite{rfc9260,rfc6458,rfc3758,rfc8260}.
\item Runtime integration is technically modest when implemented through existing socket abstraction seams.
\item Real interoperability with C++ Linux SCTP endpoints demonstrates practical viability.
\end{enumerate}

The remainder of this paper builds this argument from standards context, implementation design, and measured evaluation.
