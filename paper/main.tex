\documentclass[conference]{IEEEtran}

\usepackage{cite}
\usepackage{amsmath,amssymb}
\usepackage{graphicx}
\usepackage{booktabs}
\usepackage{siunitx}
\usepackage{url}

\title{SCTP as a First-Class Transport in Go on Linux:\\
An Implementation and Interoperability Study}

\author{
\IEEEauthorblockN{Olivier Van Acker}
\IEEEauthorblockA{
Independent Researcher\\
London, United Kingdom\\
olivier@robotmotel.com}
}

\begin{document}

\maketitle

\begin{abstract}
This paper revisits a 2013 SCTP-in-Go implementation and re-establishes the design on Linux in a modern Go runtime tree. The central claim is that SCTP should be treated as a first-class transport protocol next to TCP and UDP, not as an exotic add-on. The argument is architectural and empirical: Internet protocol layering was designed to evolve, SCTP is a mature IETF transport, and practical integration in a contemporary runtime is straightforward when the networking stack is already protocol-agnostic at key extension points. We describe the Linux integration in Go's \texttt{net} package, show API parity with existing TCP/UDP patterns, and evaluate interoperability against an independent C++ endpoint built on Linux SCTP APIs. Results show that the implementation passes in-tree SCTP tests and bilateral Go<->C++ interop scenarios while preserving SCTP-specific metadata such as stream identifier and PPID. The outcome supports a broader position: modern language runtimes should expose SCTP as a standard transport option for message-oriented and resilience-aware applications rather than forcing all designs through TCP or UDP.
\end{abstract}

\begin{IEEEkeywords}
SCTP, Go runtime, Linux networking, transport protocols, interoperability, sockets API
\end{IEEEkeywords}

\section{Introduction}
\label{sec:intro}

Transport protocol choice in many software stacks has narrowed to TCP or UDP, even though Internet architecture never required such a binary choice. The original internetworking design emphasized extensibility and evolution at protocol boundaries \cite{cerf1974protocol,rfc1122,rfc1958}. SCTP was standardized to provide reliable, congestion-controlled, message-oriented transport with features that are difficult or expensive to emulate correctly above TCP or UDP \cite{rfc9260,rfc6458,stewart2001sctp}.

This paper revisits an earlier 2013 implementation effort and updates the work for Linux and a modern Go runtime source tree. The goal is not to present SCTP as ``new'', but to demonstrate that it can be integrated into a mainstream language runtime with the same first-class treatment given to TCP and UDP. In this context, first-class means:

\begin{itemize}
\item explicit address and connection types in the standard networking package,
\item normal resolver, dial, and listen flows,
\item direct support for protocol-specific metadata and controls,
\item test coverage and interoperability validation against an independent implementation.
\end{itemize}

The implementation in this repository adds SCTP support directly inside Go's \texttt{net} package on Linux, including public API surface (\texttt{SCTPAddr}, \texttt{SCTPConn}, \texttt{DialSCTP}, \texttt{ListenSCTP}), Linux data-path support for \texttt{sendmsg}/\texttt{recvmsg}, and targeted test and interop harnesses.

The paper's central thesis is that SCTP belongs next to TCP and UDP in modern runtime APIs because:

\begin{enumerate}
\item Internet protocol architecture is intentionally extensible \cite{rfc1122,rfc1958}.
\item SCTP is a mature standards-track transport with broad, production-relevant semantics \cite{rfc9260,rfc6458,rfc3758,rfc8260}.
\item Runtime integration is technically modest when implemented through existing socket abstraction seams.
\item Real interoperability with C++ Linux SCTP endpoints demonstrates practical viability.
\end{enumerate}

The remainder of this paper builds this argument from standards context, implementation design, and measured evaluation.

\section{Why SCTP Should Be First-Class}
\label{sec:why}

\subsection{Architectural Basis: IP Was Designed to Evolve}
The Internet stack has long separated network and transport concerns to permit independent evolution \cite{cerf1974protocol,rfc791,rfc1122}. RFC 1958 makes this explicit: successful protocol design assumes change and adaptation over time \cite{rfc1958}. Treating TCP and UDP as immutable endpoints of transport evolution is therefore an operational habit, not an architectural law.

\subsection{SCTP Is Not Experimental}
SCTP is a standards-track transport protocol with a mature socket API and long-running implementations \cite{rfc9260,rfc6458,linuxsctp7}. Relative to canonical TCP and UDP definitions \cite{rfc9293,rfc768}, it combines properties that many applications need:

\begin{itemize}
\item reliable, congestion-aware transport like TCP,
\item preservation of message boundaries unlike TCP,
\item multistreaming to reduce head-of-line coupling,
\item optional multihoming and failover support at transport layer,
\item protocol-level extensibility (e.g., partial reliability, schedulers).
\end{itemize}

These are not theoretical features; they are codified in the base protocol and extensions \cite{rfc3758,rfc5061,rfc7053,rfc8260}.

\subsection{Industry Relevance}
SCTP is also used as the transport substrate for WebRTC data channels \cite{rfc8831,rfc8832}. This matters for two reasons: first, it disproves the claim that SCTP has no mainstream deployment; second, it shows that modern systems already depend on SCTP semantics when message orientation and stream multiplexing are required.

\subsection{Comparison with TCP and UDP}
Table \ref{tab:transport-compare} summarizes why SCTP deserves parity in language runtime APIs.

\begin{table}[htbp]
\caption{Transport Feature Comparison}
\label{tab:transport-compare}
\centering
\begin{tabular}{lccc}
\toprule
Feature & TCP & UDP & SCTP \\
\midrule
Reliable delivery & Yes & No & Yes \\
Message boundaries & No & Yes & Yes \\
Built-in multistreaming & No & No & Yes \\
Transport multihoming & No & No & Yes \\
Congestion control & Yes & No & Yes \\
Socket API standardization & Yes & Yes & Yes \\
\bottomrule
\end{tabular}
\end{table}

The practical conclusion is straightforward: if TCP and UDP are exposed natively in a runtime's core networking API, SCTP can be justified by the same criterion used for those protocols, namely broad utility plus standards-based interoperability.

\section{Design and Integration in the Go Runtime}
\label{sec:design}

\subsection{Integration Goal}
The implementation target is parity with existing \texttt{net} package transport patterns: address resolution, dial/listen entry points, and connection methods should look and behave like TCP/UDP equivalents while exposing SCTP-specific controls where required.

\subsection{Public API Additions}
The Linux implementation introduces SCTP-specific types and functions in \texttt{src/net/sctpsock.go}, including:

\begin{itemize}
\item \texttt{type SCTPAddr \{ IP net.IP; Port int; Zone string \}}
\item \texttt{ResolveSCTPAddr(network, address)}
\item \texttt{DialSCTP(network, laddr, raddr)}
\item \texttt{ListenSCTP(network, laddr)}
\item \texttt{type SCTPConn} with \texttt{ReadFromSCTP}, \texttt{WriteToSCTP}, \texttt{SetNoDelay}, \texttt{SetInitOptions}, \texttt{SubscribeEvents}
\end{itemize}

These additions mirror established Go networking style and maintain compatibility with common \texttt{Conn}/\texttt{PacketConn} usage.

\subsection{Runtime Touchpoints}
Implementation is localized to existing extensibility seams in \texttt{net}:

\begin{itemize}
\item Resolver and network parsing integration in \texttt{src/net/ipsock.go}
\item Dial/listen dispatch integration in \texttt{src/net/dial.go}
\item Linux SCTP socket operations in \texttt{src/net/sctpsock\_posix.go}
\item Linux SCTP constants and control-message marshaling in \texttt{src/net/sctpsock\_linux.go}
\item Stubbed behavior for unsupported targets in \texttt{src/net/sctpsock\_stub.go}
\end{itemize}

This is the core evidence for integration ease: SCTP support was added by extending existing abstractions rather than redesigning the runtime networking architecture.

\subsection{Data Path and Metadata}
The implementation uses \texttt{SOCK\_SEQPACKET} with \texttt{IPPROTO\_SCTP}. Message transmission and reception are built on \texttt{sendmsg}/\texttt{recvmsg}, with ancillary control messages carrying SCTP metadata (\texttt{SCTP\_SNDINFO}, \texttt{SCTP\_RCVINFO}). This preserves message orientation while exposing stream and PPID metadata to applications.

\subsection{One-to-Many First}
The current design emphasizes one-to-many SCTP semantics for Linux. \texttt{DialSCTP} uses an unconnected socket model with remote destination retained for default sends, and \texttt{ListenSCTP} follows passive receive behavior on \texttt{SOCK\_SEQPACKET}. This keeps the API simple while supporting key SCTP behavior.

\begin{figure}[htbp]
\centering
\fbox{
\parbox{0.47\textwidth}{
\textbf{Integration map in Go \texttt{net} package}\\
1) Parse network: \texttt{sctp}, \texttt{sctp4}, \texttt{sctp6}\\
2) Resolve endpoint: \texttt{SCTPAddr}\\
3) Dial/listen dispatch in shared socket path\\
4) Linux SCTP options and cmsg handlers\\
5) Public \texttt{SCTPConn} methods for metadata-aware I/O
}
}
\caption{SCTP follows existing TCP/UDP extension seams in Go networking internals.}
\label{fig:integration-map}
\end{figure}

\section{Evaluation Method}
\label{sec:method}

\subsection{Objectives}
The evaluation verifies two claims:
\begin{enumerate}
\item SCTP is integrated as a first-class runtime transport, not as an out-of-tree shim.
\item The implementation interoperates with an independent C++ SCTP endpoint on Linux.
\end{enumerate}

\subsection{Environment}
Measurements in this paper were captured on:
\begin{itemize}
\item OS: Ubuntu 25.10, Linux kernel 6.17.0-12-generic
\item Architecture: \texttt{linux/amd64}
\item Go tree: in-repo build, version \texttt{go1.27-devel\_c9cbeb0}
\item SCTP kernel module enabled (\texttt{sctp})
\end{itemize}

\subsection{Go Package Tests}
Targeted tests in \texttt{net} were executed:
\begin{itemize}
\item \texttt{TestSCTPLoopbackReadWrite}
\item \texttt{TestSCTPUnsupportedOnBadNetwork}
\item \texttt{TestResolveSCTPAddrUnknownNetwork}
\item \texttt{TestParseNetworkSCTP}
\item \texttt{TestSCTPAddrString}
\end{itemize}

These tests validate local SCTP I/O behavior, error semantics, and network/address parsing paths.

\subsection{Go $\leftrightarrow$ C++ Interoperability Matrix}
Interop uses the repository harness at \texttt{misc/sctp-interop/harness/run\_matrix.sh} and covers:
\begin{enumerate}
\item Go server receiving from C++ client.
\item C++ server receiving from Go client.
\end{enumerate}

Both scenarios verify payload integrity and SCTP metadata transport (stream ID and PPID). The C++ side uses Linux SCTP userspace API calls and \texttt{recvmsg} ancillary parsing.

\subsection{Timing Measurement}
The complete matrix runner was executed 20 times. Wall-clock duration per run was recorded from shell timestamps. This provides a lightweight integration performance indicator for repeated bring-up, cross-language exchange, and teardown under the same configuration.

\section{Results}
\label{sec:results}

\subsection{Go In-Tree SCTP Tests}
All targeted Go \texttt{net} SCTP tests passed in a single run. The test run reported successful socket creation and loopback communication on \texttt{AF\_INET}/\texttt{SOCK\_SEQPACKET}/\texttt{IPPROTO\_SCTP}.

\subsection{Interop Matrix Outcome}
Both interoperability directions passed:
\begin{itemize}
\item Go server received C++ payload with expected stream and PPID.
\item C++ server received Go payload with expected stream and PPID.
\end{itemize}

Table \ref{tab:interop-observed} summarizes the observed data-plane correctness.

\begin{table}[htbp]
\caption{Observed Go<->C++ SCTP Interoperability}
\label{tab:interop-observed}
\centering
\begin{tabular}{p{0.20\textwidth}p{0.25\textwidth}}
\toprule
Scenario & Observed output \\
\midrule
Go server <- C++ client & \texttt{stream=3 ppid=101 payload=cpp-to-go} \\
C++ server <- Go client & \texttt{stream=4 ppid=202 payload=go-to-cpp} \\
\bottomrule
\end{tabular}
\end{table}

\subsection{Repeated Matrix Runtime}
For 20 repeated runs of the full interop matrix:
\begin{itemize}
\item mean runtime: \SI{2.187037}{\second}
\item minimum runtime: \SI{2.177443}{\second}
\item maximum runtime: \SI{2.203883}{\second}
\item standard deviation: \SI{0.006289}{\second}
\end{itemize}

The low variation suggests stable harness behavior and reproducible cross-language SCTP exchange in the tested environment.

\begin{figure}[htbp]
\centering
\fbox{
\parbox{0.47\textwidth}{
\textbf{20-run matrix timing summary}\\
Mean: 2.187 s\\
Min/Max: 2.177 s / 2.204 s\\
Std. dev.: 0.006 s\\
Data source: \texttt{paper/repro/data/interop\_matrix\_runtime\_2026-02-14.csv}
}
}
\caption{Runtime stability of the Go<->C++ SCTP matrix on Linux.}
\label{fig:runtime-summary}
\end{figure}

\subsection{Interpretation}
The results support the implementation claim:
\begin{enumerate}
\item SCTP behaves as an ordinary transport option inside Go's standard networking paths.
\item SCTP-specific metadata survives end-to-end across independent implementations.
\item Integration and validation can be automated with the same workflow style used for other runtime networking features.
\end{enumerate}

\section{Related Work}
\label{sec:related}

The original SCTP motivation and protocol model are well documented in standards and reference texts \cite{rfc9260,stewart2001sctp}. The socket API formalization in RFC 6458 is directly relevant to runtime and language binding design \cite{rfc6458}. Extensions such as partial reliability and stream scheduling further differentiate SCTP from TCP-centric abstractions \cite{rfc3758,rfc8260}.

Several prior studies evaluated SCTP capabilities. Iyengar et al. explored concurrent multipath transfer over SCTP multihoming \cite{iyengar2006cmt}. Penoff et al. described a portable userspace SCTP stack and portability/performance trade-offs \cite{penoff2012userspacesctp}. These works support the view that SCTP is both implementable and practically useful across deployment contexts.

More generally, transport evolution research has shown how deployment friction can ossify protocol diversity \cite{honda2011extendtcp}. This paper contributes from a runtime-engineering perspective: language-level first-class APIs materially reduce that friction by making non-TCP transports available through familiar interfaces.

Finally, WebRTC data channels demonstrate broad production exposure of SCTP semantics in modern applications \cite{rfc8831,rfc8832}. This real-world adoption weakens arguments that SCTP is niche or obsolete.

\section{Discussion}
\label{sec:discussion}

\subsection{Why This Supports ``First-Class'' Status}
The implementation and evaluation provide concrete evidence for first-class SCTP treatment in a modern runtime:

\begin{itemize}
\item \textbf{API symmetry:} SCTP follows the same naming and usage patterns as TCP/UDP in Go's \texttt{net} package.
\item \textbf{Internal symmetry:} SCTP plugs into existing resolver, dial, listen, and file-descriptor workflows.
\item \textbf{Interoperability:} wire-level behavior is compatible with independent Linux C++ SCTP endpoints.
\item \textbf{Operational symmetry:} tests and harness execution integrate into standard CI-style workflows.
\end{itemize}

This is precisely what ``first-class'' should mean in a runtime context.

\subsection{The Extensibility Argument}
Internet architecture intentionally supports protocol evolution \cite{cerf1974protocol,rfc1122,rfc1958}. Treating SCTP as a default-capable option is consistent with this model and avoids unnecessary application-layer reimplementation of transport semantics. This is aligned with end-to-end design reasoning: semantics that belong at transport should not be rebuilt repeatedly in each application \cite{saltzer1984endtoend}. In practice, forcing message-oriented or multi-stream designs onto TCP often reintroduces complexity that SCTP already standardizes.

\subsection{Additional Practical Arguments}
Beyond architectural correctness, several practical arguments support broader SCTP exposure:
\begin{itemize}
\item \textbf{Correctness:} protocol-level message framing reduces application parsing ambiguity.
\item \textbf{Performance isolation:} multistreaming can reduce cross-flow head-of-line coupling.
\item \textbf{Resilience:} multihoming and association semantics improve failover options.
\item \textbf{Security posture:} SCTP's association handshake and extension mechanisms provide robust baseline behavior \cite{rfc9260}.
\end{itemize}

None of these claims require replacing TCP/UDP; they require giving developers a native, standards-based choice.

\section{Limitations and Threats to Validity}
\label{sec:limits}

This study has clear scope boundaries:

\begin{itemize}
\item Evaluation was performed on one Linux environment and loopback/local-host paths.
\item Interop matrix validates bidirectional Go $\leftrightarrow$ C++ exchange, but does not exhaust all SCTP extensions.
\item The current harness does not benchmark high-throughput bulk transfer against tuned TCP/UDP baselines.
\item Multihoming failover scenarios were not fully exercised in a multi-interface fault-injection lab.
\end{itemize}

These limitations do not invalidate the central result (integration feasibility plus interop correctness), but they constrain generalization of quantitative performance conclusions.

Future work should include multi-host and multi-path experiments, scheduler/interleaving extension coverage, and comparative workload benchmarks for message-centric applications.

\section{Conclusion}
\label{sec:conclusion}

This paper presented a Linux-focused SCTP integration into a modern Go runtime and evaluated it with in-tree tests plus Go<->C++ interoperability scenarios. The evidence supports the main claim: SCTP can and should be treated as a first-class transport next to TCP and UDP in runtime networking APIs.

The broader point is architectural. Internet protocol design expects extensibility, and SCTP is a mature transport standard that addresses real application needs without forcing ad hoc reimplementation at higher layers. The implementation in this repository demonstrates that enabling SCTP in a mainstream runtime is not a disruptive redesign; it is an incremental, testable extension of existing networking abstractions.

Treating SCTP as first-class is therefore both technically justified and operationally practical.


\bibliographystyle{IEEEtran}
\bibliography{refs}

\end{document}
